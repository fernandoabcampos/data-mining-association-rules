\documentclass[]{article}
\usepackage{lmodern}
\usepackage{amssymb,amsmath}
\usepackage{ifxetex,ifluatex}
\usepackage{fixltx2e} % provides \textsubscript
\ifnum 0\ifxetex 1\fi\ifluatex 1\fi=0 % if pdftex
  \usepackage[T1]{fontenc}
  \usepackage[utf8]{inputenc}
\else % if luatex or xelatex
  \ifxetex
    \usepackage{mathspec}
  \else
    \usepackage{fontspec}
  \fi
  \defaultfontfeatures{Ligatures=TeX,Scale=MatchLowercase}
\fi
% use upquote if available, for straight quotes in verbatim environments
\IfFileExists{upquote.sty}{\usepackage{upquote}}{}
% use microtype if available
\IfFileExists{microtype.sty}{%
\usepackage{microtype}
\UseMicrotypeSet[protrusion]{basicmath} % disable protrusion for tt fonts
}{}
\usepackage[margin=1in]{geometry}
\usepackage{hyperref}
\hypersetup{unicode=true,
            pdftitle={PEC 5 - Reglas de Asociación},
            pdfauthor={Fernando Antonio Barbeiro Campos - fbarbeiro@uoc.edu},
            pdfborder={0 0 0},
            breaklinks=true}
\urlstyle{same}  % don't use monospace font for urls
\usepackage{color}
\usepackage{fancyvrb}
\newcommand{\VerbBar}{|}
\newcommand{\VERB}{\Verb[commandchars=\\\{\}]}
\DefineVerbatimEnvironment{Highlighting}{Verbatim}{commandchars=\\\{\}}
% Add ',fontsize=\small' for more characters per line
\usepackage{framed}
\definecolor{shadecolor}{RGB}{248,248,248}
\newenvironment{Shaded}{\begin{snugshade}}{\end{snugshade}}
\newcommand{\KeywordTok}[1]{\textcolor[rgb]{0.13,0.29,0.53}{\textbf{#1}}}
\newcommand{\DataTypeTok}[1]{\textcolor[rgb]{0.13,0.29,0.53}{#1}}
\newcommand{\DecValTok}[1]{\textcolor[rgb]{0.00,0.00,0.81}{#1}}
\newcommand{\BaseNTok}[1]{\textcolor[rgb]{0.00,0.00,0.81}{#1}}
\newcommand{\FloatTok}[1]{\textcolor[rgb]{0.00,0.00,0.81}{#1}}
\newcommand{\ConstantTok}[1]{\textcolor[rgb]{0.00,0.00,0.00}{#1}}
\newcommand{\CharTok}[1]{\textcolor[rgb]{0.31,0.60,0.02}{#1}}
\newcommand{\SpecialCharTok}[1]{\textcolor[rgb]{0.00,0.00,0.00}{#1}}
\newcommand{\StringTok}[1]{\textcolor[rgb]{0.31,0.60,0.02}{#1}}
\newcommand{\VerbatimStringTok}[1]{\textcolor[rgb]{0.31,0.60,0.02}{#1}}
\newcommand{\SpecialStringTok}[1]{\textcolor[rgb]{0.31,0.60,0.02}{#1}}
\newcommand{\ImportTok}[1]{#1}
\newcommand{\CommentTok}[1]{\textcolor[rgb]{0.56,0.35,0.01}{\textit{#1}}}
\newcommand{\DocumentationTok}[1]{\textcolor[rgb]{0.56,0.35,0.01}{\textbf{\textit{#1}}}}
\newcommand{\AnnotationTok}[1]{\textcolor[rgb]{0.56,0.35,0.01}{\textbf{\textit{#1}}}}
\newcommand{\CommentVarTok}[1]{\textcolor[rgb]{0.56,0.35,0.01}{\textbf{\textit{#1}}}}
\newcommand{\OtherTok}[1]{\textcolor[rgb]{0.56,0.35,0.01}{#1}}
\newcommand{\FunctionTok}[1]{\textcolor[rgb]{0.00,0.00,0.00}{#1}}
\newcommand{\VariableTok}[1]{\textcolor[rgb]{0.00,0.00,0.00}{#1}}
\newcommand{\ControlFlowTok}[1]{\textcolor[rgb]{0.13,0.29,0.53}{\textbf{#1}}}
\newcommand{\OperatorTok}[1]{\textcolor[rgb]{0.81,0.36,0.00}{\textbf{#1}}}
\newcommand{\BuiltInTok}[1]{#1}
\newcommand{\ExtensionTok}[1]{#1}
\newcommand{\PreprocessorTok}[1]{\textcolor[rgb]{0.56,0.35,0.01}{\textit{#1}}}
\newcommand{\AttributeTok}[1]{\textcolor[rgb]{0.77,0.63,0.00}{#1}}
\newcommand{\RegionMarkerTok}[1]{#1}
\newcommand{\InformationTok}[1]{\textcolor[rgb]{0.56,0.35,0.01}{\textbf{\textit{#1}}}}
\newcommand{\WarningTok}[1]{\textcolor[rgb]{0.56,0.35,0.01}{\textbf{\textit{#1}}}}
\newcommand{\AlertTok}[1]{\textcolor[rgb]{0.94,0.16,0.16}{#1}}
\newcommand{\ErrorTok}[1]{\textcolor[rgb]{0.64,0.00,0.00}{\textbf{#1}}}
\newcommand{\NormalTok}[1]{#1}
\usepackage{longtable,booktabs}
\usepackage{graphicx,grffile}
\makeatletter
\def\maxwidth{\ifdim\Gin@nat@width>\linewidth\linewidth\else\Gin@nat@width\fi}
\def\maxheight{\ifdim\Gin@nat@height>\textheight\textheight\else\Gin@nat@height\fi}
\makeatother
% Scale images if necessary, so that they will not overflow the page
% margins by default, and it is still possible to overwrite the defaults
% using explicit options in \includegraphics[width, height, ...]{}
\setkeys{Gin}{width=\maxwidth,height=\maxheight,keepaspectratio}
\IfFileExists{parskip.sty}{%
\usepackage{parskip}
}{% else
\setlength{\parindent}{0pt}
\setlength{\parskip}{6pt plus 2pt minus 1pt}
}
\setlength{\emergencystretch}{3em}  % prevent overfull lines
\providecommand{\tightlist}{%
  \setlength{\itemsep}{0pt}\setlength{\parskip}{0pt}}
\setcounter{secnumdepth}{0}
% Redefines (sub)paragraphs to behave more like sections
\ifx\paragraph\undefined\else
\let\oldparagraph\paragraph
\renewcommand{\paragraph}[1]{\oldparagraph{#1}\mbox{}}
\fi
\ifx\subparagraph\undefined\else
\let\oldsubparagraph\subparagraph
\renewcommand{\subparagraph}[1]{\oldsubparagraph{#1}\mbox{}}
\fi

%%% Use protect on footnotes to avoid problems with footnotes in titles
\let\rmarkdownfootnote\footnote%
\def\footnote{\protect\rmarkdownfootnote}

%%% Change title format to be more compact
\usepackage{titling}

% Create subtitle command for use in maketitle
\newcommand{\subtitle}[1]{
  \posttitle{
    \begin{center}\large#1\end{center}
    }
}

\setlength{\droptitle}{-2em}
  \title{PEC 5 - Reglas de Asociación}
  \pretitle{\vspace{\droptitle}\centering\huge}
  \posttitle{\par}
  \author{Fernando Antonio Barbeiro Campos -
\href{mailto:fbarbeiro@uoc.edu}{\nolinkurl{fbarbeiro@uoc.edu}}}
  \preauthor{\centering\large\emph}
  \postauthor{\par}
  \predate{\centering\large\emph}
  \postdate{\par}
  \date{5 de May, 2018}


\begin{document}
\maketitle

\section{Presentación}\label{presentacion}

Esta práctica cubre los Módulos 6 y 8 (Evaluación de modelos) del
programa de la asignatura.

\section{Competencias}\label{competencias}

Las competencias que se trabajan en esta prueba son:

\begin{itemize}
\item
  Uso y aplicación de las TIC en el ámbito académico y profesional.
\item
  Capacidad para innovar y generar nuevas ideas.
\item
  Capacidad para evaluar soluciones tecnológicas y elaborar propuestas
  de proyectos teniendo en cuenta los recursos, las alternativas
  disponibles y las condiciones de mercado.
\item
  Conocer las tecnologías de comunicaciones actuales y emergentes, así
  como saberlas aplicar convenientemente para diseñar y desarrollar
  soluciones basadas en sistemas y tecnologías de la información.
\item
  Aplicación de las técnicas específicas de ingeniería del software en
  las diferentes etapas del ciclo de vida de un proyecto.
\item
  Capacidad para aplicar las técnicas específicas de tratamiento,
  almacenamiento y administración de datos.
\item
  Capacidad para proponer y evaluar diferentes alternativas tecnológicas
  para resolver un problema concreto.
\end{itemize}

\section{Objetivos}\label{objetivos}

La correcta asimilación del Módulo 6 y el resto de Módulos trabajados:
En esta PEC trabajaremos la generación e interpretación de un modelo de
basado en reglas de asociación con los recursos de prácticas.

\section{Enunciado}\label{enunciado}

Contextualizad los ejemplos de las siguientes preguntas respecto al
proyecto que has definido en la PEC1. Si lo deseáis, podéis redefinir o
afinar el proyecto.

\begin{enumerate}
\def\labelenumi{\arabic{enumi}.}
\tightlist
\item
  ¿creéis que las reglas de asociación podrían ser el método finalmente
  escogido? ¿Os podrían aportar alguna cosa?

  \begin{itemize}
  \tightlist
  \item
    ¿Cómo podría ser el modelo resultante?
  \item
    Dar un ejemplo de la interpretación que se podría derivar del modelo
    generado
  \end{itemize}
\end{enumerate}

\subsection{Respuesta}\label{respuesta}

Así como ya había Absolutamente los métodos de agregación no son los más
adecuados para elegir en el caso de baggage-propensity. La justificativa
es que, conforme hemos comentado en la PEC3, tratase de un modelo de
aprendizaje supervisado, esto es, conocemos a priori las categorías
(labels). Mientras tanto, modelos de agregación (que dan como resultado
modelos descriptivos) buscan obtener una primera aproximación con
relación al dominio de la información, o sea, son modelos de aprendizaje
no supervisados.

\begin{longtable}[]{@{}ll@{}}
\toprule
First Header & Second Header\tabularnewline
\midrule
\endhead
Content Cell & Content Cell\tabularnewline
Content Cell & Content Cell\tabularnewline
\bottomrule
\end{longtable}

\begin{Shaded}
\begin{Highlighting}[]
\CommentTok{#my_file <- "lastfm.csv"}
\CommentTok{#cat("\textbackslash{}n", file = my_file, append = TRUE)}
\CommentTok{#tdata <- read.transactions(file = my_file, rm.duplicates = FALSE, skip = 1, sep = ",")}
\CommentTok{#head(tdata)}
\CommentTok{#class(tdata)}
\CommentTok{#inspect(head(tdata))}
\CommentTok{#size(head(tdata)) }

\CommentTok{#frequentItems <- eclat (tdata, parameter = list(supp = 0.07, maxlen = 15))}
\CommentTok{#inspect(frequentItems)}
\CommentTok{#itemFrequencyPlot(tdata, topN=15, type="absolute", main="Item Frequency") }
\end{Highlighting}
\end{Shaded}

\begin{Shaded}
\begin{Highlighting}[]
\CommentTok{#rules <- apriori (tdata, parameter = list(supp = 0.001, conf = 0.5))}
\CommentTok{#rules_conf <- sort (rules, by="confidence", decreasing=TRUE)}
\CommentTok{#inspect(head(rules_conf))}
\CommentTok{#rules_lift <- sort (rules, by="lift", decreasing=TRUE) }
\CommentTok{#inspect(head(rules_lift))}
\NormalTok{lastfm <-}\StringTok{ }\KeywordTok{read.csv}\NormalTok{(}\DataTypeTok{file =} \StringTok{"lastfm.csv"}\NormalTok{)}
\CommentTok{#Distribucion de valores}
\KeywordTok{head}\NormalTok{(}\KeywordTok{table}\NormalTok{(lastfm}\OperatorTok{$}\NormalTok{artist))}
\end{Highlighting}
\end{Shaded}

\begin{verbatim}
## 
## ...and you will know us by the trail of dead 
##                                          147 
##                                    [unknown] 
##                                          553 
##                                         2pac 
##                                          341 
##                                 3 doors down 
##                                          464 
##                           30 seconds to mars 
##                                          492 
##                                          311 
##                                          125
\end{verbatim}

\begin{Shaded}
\begin{Highlighting}[]
\KeywordTok{head}\NormalTok{(}\KeywordTok{table}\NormalTok{(lastfm}\OperatorTok{$}\NormalTok{user))}
\end{Highlighting}
\end{Shaded}

\begin{verbatim}
## 
##  1  3  4  5  6  7 
## 16 29 27 11 23 22
\end{verbatim}

\begin{Shaded}
\begin{Highlighting}[]
\KeywordTok{table}\NormalTok{(lastfm}\OperatorTok{$}\NormalTok{sex)}
\end{Highlighting}
\end{Shaded}

\begin{verbatim}
## 
##      f      m 
##  78132 211823
\end{verbatim}

\begin{Shaded}
\begin{Highlighting}[]
\KeywordTok{head}\NormalTok{(}\KeywordTok{table}\NormalTok{(lastfm}\OperatorTok{$}\NormalTok{country))}
\end{Highlighting}
\end{Shaded}

\begin{verbatim}
## 
##    Afghanistan        Albania        Algeria American Samoa        Andorra 
##             51             49             42             16             33 
##         Angola 
##             38
\end{verbatim}


\end{document}
